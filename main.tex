\documentclass[12pt,a4paper]{article}

% --- Pakete ---
\usepackage[utf8]{inputenc}
\usepackage[T1]{fontenc}
\usepackage[ngerman]{babel}
\usepackage{lmodern}
\usepackage{geometry}
\usepackage{setspace}
\usepackage{graphicx}
\usepackage{csquotes}
\usepackage{amsmath, amssymb}
\usepackage{fancyhdr}
\usepackage{titlesec}
\usepackage{hyperref}

% --- Seitenränder ---
\geometry{a4paper, top=3cm, bottom=3cm, left=3.5cm, right=2.5cm}

% --- Zeilenabstand ---
\onehalfspacing

% --- Kopf-/Fußzeile ---
\pagestyle{fancy}
\fancyhf{}
\fancyhead[L]{Seminararbeit}
\fancyhead[R]{\thepage}

% --- Titelinformationen ---
\title{Titel der Seminararbeit}
\author{Max Mustermann \\
Matrikelnummer: 1234567 \\
Universität Musterstadt \\
Fachbereich: Beispielwissenschaft}
\date{\today}

\begin{document}

% --- Titelseite ---
\begin{titlepage}
    \centering
    \vspace*{3cm}
    {\LARGE\bfseries Titel der Seminararbeit \par}
    \vspace{2cm}
    {\large Max Mustermann \par}
    \vspace{0.5cm}
    Matrikelnummer: 1234567 \par
    Fachbereich: Beispielwissenschaft \par
    Universität Musterstadt \par
    \vfill
    \today
\end{titlepage}

% --- Inhaltsverzeichnis ---
\tableofcontents
\newpage

% --- Einleitung ---
\section{Einleitung}
Dies ist ein Beispieltext für eine Seminararbeit. In der Einleitung wird das Thema eingeführt und die Zielsetzung der Arbeit erläutert. \par
„Dies ist ein Zitat als Beispiel.“\footnote{Max Beispiel, *Beispielhafte Quellenangabe*, 2023.}

% --- Hauptteil ---
\section{Theoretischer Hintergrund}
Hier folgt eine theoretische Erläuterung des Themas. Mathematische Formeln können wie folgt eingefügt werden:
\[
    E = mc^2
\]
Auch Abbildungen lassen sich einfügen:

\begin{figure}[h!]
    \centering
    \includegraphics[width=0.4\textwidth]{example-image}
    \caption{Beispielabbildung}
\end{figure}

\section{Analyse und Diskussion}
Hier kann die Analyse stattfinden. Zum Beispiel:
\begin{itemize}
    \item Erste Beobachtung
    \item Zweite Beobachtung
    \item Dritte Beobachtung
\end{itemize}

% --- Fazit ---
\section{Fazit}
Das Fazit fasst die wichtigsten Erkenntnisse zusammen und gibt ggf. einen Ausblick.

% --- Literaturverzeichnis ---
\newpage
\section*{Literaturverzeichnis}
\begin{itemize}
    \item Max Beispiel: \textit{Beispielhafte Quellenangabe}. Musterverlag, 2023.
\end{itemize}

\end{document}

